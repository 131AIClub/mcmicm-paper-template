
% 开始写 memo 信
% 更换字体为 palatino 也可以不换
\setmainfont{texgyrepagella-regular.otf} % linux 字体设置
% \setmainfont{TeX Gyre Pagella} % windows 字体设置
\newpage
\newgeometry{left = 3.5cm, right = 3.5cm}
\thispagestyle{empty}

{\centering \fontsize{18pt}{14pt}\selectfont \textbf{MEMO}\par}

\noindent FROM: Team {} 1917694 , MCM C

\noindent To: The group of Governors

\noindent Date: January 28, 2019

\vspace{10pt}

Dear Officials:

It is our honor to help you with analyzing the spread and characteristics of opioid. We are writing this letter to report our findings.

We analyzed the characteristics of opioid spread in the five states and counties, and found that if left uncontrolled, the opioid abuse will continue to spread, and the number of drug users will increase. We also found that there is no obvious relationship between the spread of opioid use and the geographical location. We deduced that Ohio is most likely the originated state of opioid use, and the drug cases will increase significantly over time. Of all five states, West Virginia shows a relatively healthy status. We also analyzed each county in the five states. Taking West Virginia as an example, Monongalia and Berkeley may be the originating counties for heroin transmission, and drug cases in these two counties are mainly spread to Kanawha and Berkeley.

We determined two epidemic thresholds based on characteristics of opioid spread, with respect to the number of opioid cases and opioid spread rate respectively. We found that the number of narcotic analgesics cases in Ohio reaches the threshold in 2022, and the number of heroin cases in Pennsylvania reached the threshold in 2023.

We analyzed the correlations between the number of opioid cases and various socio-economic factors. We found that the number of opioids cases shows high correlations with disability status, educational level, family status and adolescents. Therefore, we provide suggestions based on these factors to help suppress opioid spread. We also evaluate the effectiveness of these strategies. The result shows that when the amount of drug spread is less than 26.3\% on before, the number of drug users will be significantly reduced in 3 years, and become lower than that of 2010 in 5 years. 

We also do sensitivity analysis of our model. We found that inflows of drugs from outside the states also plays an important role in affecting the number of opioid cases.

Based on our analysis, we provide some suggestions to effectively prevent opioid spread:

1. National and local governments play an important role in detecting and preventing opioid overdose and abuse. Monitoring systems and early warning mechanisms should be established to effectively detect and respond to the spread of opioid abuse.

2. Provide more job opportunities to alleviate people's pressure, even to set up a free subsidy program. The local government should also publicize the harm of opioids abuse.

3. Prescriptions and sales of opioids should be strictly controlled, and a feedback mechanism for mandatory postoperative hospital visits should be established. The patients should also be examined for abuse of opioids during the hospital examination.

4. Impose careful check and strict management of logistics, to reduce the impact of opioids from outside the states.

We believe that our model is useful for preventing opioid spread in the future. You are welcome to contact us at any time for further cooperation.

% 信多出一页,清理页眉页脚
\thispagestyle{empty}
% 信的结尾
{\raggedleft
Sincerely yours

MCM C Team 1917694\par
}